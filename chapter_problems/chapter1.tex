\chapter{The Real and Complex Number Systems}
\label{ch1}

\begin{example}
\item
    Prove that there is no rational \(p\) such that \(p^2 = 2\).
    \par\smallskip
    Suppose \(p\) is rational. Then \(p = m/n\) where \(m, n\in\mathbb{Z}\) such that both \(m\) and \(n\) are not 
    even.
    \begin{alignat}{3}
    	p^2 &= \frac{m^2}{n^2} &&= 2\notag\\
    	&= m^2 &&= 2n^2 \label{eq:1.1}
    \end{alignat}
    Therefore, since \(m^2 = 2n^2\), \(m^2\) is even and \(m\) is even. Since \(m\) is even, we can write \(m = 2a\) so
    \(m^2 = 4a^2\) and \(m^2\) is divisible by \(4\). Now that the left hand side of \cref{eq:1.1} is divisible by 
    \(4\) so is the right hand side. Thus, we have reached a contradiction since this would imply both \(m\) and \(n\) 
    are even.
    \par\smallskip
    Another common way to prove this is to use the Fundamental Theorem of Arithmetic.
    \begin{Theorems}{Fundamental Theorem of Arithmetic}{}
    	Also known as the Unique Prime Factorization Theorem states that for every \(z\in\mathbb{Z}^+\) either is prime
    	or can be expressed as the product of prime numbers and that this representation is unique up to the order of
    	the factors.
    \end{Theorems}
    Again, suppose \(p = m / n\) where \(m, n\in\mathbb{Z}\) are co-prime. From \cref{eq:1.1}, we have
    \[
    	2n^2 = m^2.
    \]
    By the FTA, both \(n\) and \(m\) can be expressed as unique product of prime factors. Let \(p_i\) be the ith prime 
    for \(n\) and \(q_i\) be the ith prime for \(b\) where \(\alpha_i\) is ith power of \(n\) and \(\beta_i\) is the ith power 
    \(m\). Then
    \begin{align}
    	n^2 &= p_1^{2\alpha_1}\cdot  p_2^{2\alpha_2}\cdots  p_k^{2\alpha_k}\notag\\
    	m^2 &= q_1^{2\beta_1}\cdot  q_2^{2\beta_2}\cdots  q_l^{2\beta_l}\notag
    \end{align}
    Let \(p_j\) and \(q_j\) be the representation of prime factorization of \(2\). Then we have
    \begin{alignat}{2}
    	2\cdot 2^{2\alpha_j} &= 2^{2\beta_j}\notag\\
    	2^{2\alpha_j + 1} &= 2^{2\beta_j}\label{eq:1.2}
    \end{alignat}
    From \cref{eq:1.2}, we have \(2\alpha_j + 1\) must equal \(2\beta_j\) but this cannot be the case since 
    \(2\alpha_j + 1\) is clearly odd and \(2\beta_j\) is even. Thus, we have reached a contradiction and \(p\) is 
    irrational.
\item[1.9]
	Let \(A\) be the set of all positive of all positive rationals \(p\) such that \(p^2 < 2\) and let \(B\) consist of all 
	positive rationals \(p\) such that \(p^2 > 2\).
	\begin{enumerate}[label=\alph*.]
	\item
		Consider the sets \(A\) and \(B\) as subsets of the ordered set \(\mathbb{Q}\). The set \(A\) is bounded above. 
		In fact, the upper bounds of \(A\) are exactly the members of \(B\). Since \(B\) contains no smallest member,
		\(A\) has \textit{no least upper bound in} \(\mathbb{Q}\).
		\par\smallskip
		Similarly, \(B\) is bounded below: The set of all lower bounds of \(B\) consists of \(A\) and of all 
		\(r\in\mathbb{Q}\) with \(r\leq 0\). Since \(A\) has no largest member, \(B\) 
		\textit{has no greatest lower bound in} \(\mathbb{Q}\).
	\item
		If \(\alpha = \sup E\) exists, then \(\alpha\) may or may not be a member of \(E\). For instance, let \(E_1\) be
		the set of all \(r\in\mathbb{Q}\) with \(r < 0\). Let \(E_2\) be the set of all \(r\in\mathbb{Q}\) with \(r\leq 0\).
		Then
		\[
			\sup E_1 = \sup E_2 = 0,
		\]
		and \(0\not\in E_1\), \(0\in E_2\).
	\item
		Let \(E\) consist of all numbers \(1 / n\), where \(n = 1, 2, 3, \ldots\). Then \(\sup E = 1\), which is in \(E\), and
		\(\inf E = 0\), which is not in \(E\).
	\end{enumerate}
\end{example}