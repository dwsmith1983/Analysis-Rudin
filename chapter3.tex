\chapter{Numerical Sequences and Series}

\begin{enumerate}
\item
  Prove that convergence of \(\{s_n\}\) implies convergence of
  \(\{\lvert s_n\rvert\}\).
  Is the converse true?
  \par\smallskip
  Since \(\{s_n\}\) converges, it is Cauchy.
  Let \(\epsilon > 0\) be given.
  There exist \(n,m > N\) such that \(\lvert s_n - s_m\rvert < \epsilon\) since
  \(\{s_n\}\) is Cauchy.
  \begin{align*}
    \lvert s_n\rvert & = \lvert s_n - s_m + s_m\rvert\\
                     & \leq \lvert s_n - s_m\rvert + \lvert s_m\rvert\\
    \lvert s_n\rvert - \lvert s_m\rvert & \leq \lvert s_n - s_m\rvert\\
    \intertext{Similarly, we can show}
    \lvert s_m\rvert - \lvert s_n\rvert & \leq \lvert s_m - s_n\rvert\\
                     & = \lvert s_n - s_m\rvert
  \end{align*}
  so
  \[
  \bigl\lvert\lvert s_n\rvert - \lvert s_m\rvert\bigr\rvert\leq
  \lvert s_n - s_m\rvert < \epsilon.
  \]
  No.
  Consider the sequence \(\{s_n\} = (-1)^n\).
  Let \(\epsilon = 1\).
  If \(\{s_n\}\) converges, it will converge to \(\pm 1\).
  WLOG assume \(s_n\to 1\).
  Let \(n > N\) such that \(n\) is odd.
  \[
  \lvert (-1)^n - 1\rvert = \lvert -1 - 1\rvert = 2\not < \epsilon
  \]
  Therefore, the sequence \(\{s_n\}\) doesn't converge.
  However, \(\{\lvert s_n\rvert\}\) does converge to \(1\).
  Let \(\epsilon > 0\) given.
  There exist an \(n > N\) such that
  \(\bigl\lvert\lvert (-1)^n\rvert - 1\bigr\rvert < \epsilon\).
  For any \(n > N\), \((-1)^n = \pm 1\) and \(\lvert \pm 1\rvert = 1\).
  \[
  \bigl\lvert\lvert (-1)^n\rvert - 1\bigr\rvert =
  \lvert 1 - 1\rvert = 0 < \epsilon
  \]
\item
  Caclulate \(\lim_{n\to\infty}\sqrt{n^2 + n} - n\).
  \begin{align*}
    \lim_{n\to\infty}\sqrt{n^2 + n} - n
    & = \lim_{n\to\infty}\Bigl(\sqrt{n^2 + n} - n\Bigr)
      \frac{\sqrt{n^2 + n} + n}{\sqrt{n^2 + n} + n}\\
    & = \lim_{n\to\infty}\frac{n}{\sqrt{n^2 + n} + n}\\
    & = \lim_{n\to\infty}\frac{n}{\sqrt{n^2 + n} + n}\frac{1/n}{1/n}\\
    & = \lim_{n\to\infty}\frac{1}{\sqrt{1 + 1/n} + 1}\\
    & = \frac{1}{2}
  \end{align*}
\item
  If \(s_1 = \sqrt{2}\) and
  \[
  s_{n + 1} = \sqrt{2 + \sqrt{s_n}}
  \]
  \(n\in\mathbb{Z}^+\) prove that \(\{s_n\}\) converges, and that \(s_n < 2\)
  for \(n\in\mathbb{Z}^+\).
  \par\smallskip
  Let \(s_{n + 1} = \sqrt{2 + \sqrt{s_n}}\) be written as
  \[
  s = \sqrt{2 + s}\Rightarrow s^2 - s - 2 = 0.
  \]
  Then \(\sqrt{2 + \sqrt{s_n}} < \sqrt{2 + s}\).
  Since we are dealing with real numbers, we are only looking for positive
  \(s\).
  \[
  s^2 - s - 2 = (s - 2)(s + 1) = 0\eqnumtag\label{3.2}
  \]
  so \(s = 2, -1\).
  Thus, \(s_n < 2\) so \(\{s_n\}\) is bounded above by two.
  Additionally, since \(s_1 = \sqrt{2}\), we have that
  \(\sqrt{2}\leq s_n < 2\).
  The parabola is concave up and symmetrical about \(s = 1/2\).
  That is, \(s\) monotonically increases from \((1/2, \infty)\) so \(\{s_n\}\)
  monotonoically increases on \(\bigl[\sqrt{2}, 2\bigr)\).
  Theorem \(3.14\) states that monotonic sequences converge if and only if it
  is bounded.
  Therefore, since \(\{s_n\}\) is bounded and monotonic, \(\{s_n\}\) converges
  and it converges to \(2\).
\item
  Find the upper and lower limits of the sequence \(\{s_n\}\) defined by
  \[
  s_1 = 0,\qquad s_{2m} = \frac{s_{2m - 1}}{2},\qquad
  s_{2m + 1} = \frac{1}{2} + s_{2m}.
  \]
  Let's determine a few of the terms.
  Then \(s_{2m} = \{0,1/4,3/8,7/16,\ldots\}\) and
  \(s_{2m + 1} = \{1/2,3/4,7/8,15/16,\ldots\}\) or we can write them as
  \begin{align*}
    s_{2m} & = \frac{1}{2} - \frac{1}{2^m}\\
    s_{2m + 1} & = 1 - \frac{1}{2^m}
  \end{align*}
  The \(\lim_{n\to\infty}\sup s_n = \lim_{n\to\infty} 1 - \frac{1}{2^n} = 1\)
  and
  \(\lim_{n\to\infty}\inf s_n = \lim_{n\to\infty} \frac{1}{2} - \frac{1}{2^n}
  = \frac{1}{2}\).
\item
  For any two real sequences \(\{a_n\}\), \(\{b_n\}\), prove that
  \[
  \lim_{n\to\infty}\sup (a_n + b_n)\leq\lim_{n\to\infty}\sup a_n +
  \lim_{n\to\infty}\sup b_n,
  \]
  provided the sum on the right is not of the form \(\infty - \infty\).
  \par\smallskip
  Let \(\lim_{n\to\infty}\sup a_n = a\), \(\lim_{n\to\infty}\sup b_n = b\), and
  \(\lim_{n\to\infty}\sup (a_n + b_n) = c\) where \(a_n + b_n = c_n\).
  Let \(\epsilon > 0\) be given.
  Then there exist \(N_1,N_2\in\mathbb{Z}^+\) such that for \(n\geq N_1\) and
  \(n\geq N_2\)
  \begin{align*}
    \lvert a_n - a\rvert & < \frac{\epsilon}{2}\\
    \lvert b_n - b\rvert & < \frac{\epsilon}{2}
  \end{align*}
  Let \(N = \max\{N_1,N_2\}\).
  Then when \(n\geq N\),
  \begin{align*}
    \lvert c_n - c\rvert & = \lvert a_n + b_n - (a + b)\rvert
                           \eqnumtag\label{3.5.a}\\
                         & \leq \lvert a_n - a\rvert + \lvert b_n - b\rvert
                           \eqnumtag\label{3.5.b}\\
                         & < \frac{\epsilon}{2} + \frac{\epsilon}{2}\\
                         & = \epsilon
  \end{align*}
  From \cref{3.5.a,3.5.b}, we have that
  \[
  \lim\sup c_n\leq\lim\sup a_n + \lim\sup b_n,
  \]
  and since \(c_n = a_n + b_n\), the identity follows.
\item
  Investigate the behavior (convergence or divergence) of \(\sum a_n\) if
  \begin{enumerate}[label = (\alph*)]
  \item
    \(a_n = \sqrt{n + 1} - \sqrt{n}\)
    \par\smallskip
    Let \(S_N\) be the \(N\)th partial sum.
    Then
    \[
    S_N = \sum_{n = 0}^N\bigl(\sqrt{n + 1} - \sqrt{n}\bigr) =
    1 + \sqrt{2} - 1 + \sqrt{3} - \sqrt{2} + \cdots + \sqrt{N} - \sqrt{N - 1}
    + \sqrt{N + 1} - \sqrt{N}
    \]
    Therefore, the \(S_N = \sqrt{N + 1}\).
    \[
    \lim_{N\to\infty}S_N = \lim_{N\to\infty}\sqrt{N + 1} = \infty
    \]
    so the series doesn't converge.
  \item
    \(a_n = \frac{\sqrt{n + 1} - \sqrt{n}}{n}\)
    \par\smallskip
    Let's re-write the series by multiplying by the conjugate.
    \[
    \sum_{n = 1}^{\infty}\frac{\sqrt{n + 1} - \sqrt{n}}{n}
    \frac{\sqrt{n + 1} + \sqrt{n}}{\sqrt{n + 1} + \sqrt{n}} =
    \sum_{n = 1}^{\infty}\frac{1}{n\bigl(\sqrt{n + 1} + \sqrt{n}\bigr)}
    \]
    Now
    \[
    \sum_{n = 1}^{\infty}\frac{1}{n\bigl(\sqrt{n + 1} + \sqrt{n}\bigr)}\leq
    \sum_{n = 1}^{\infty}\frac{1}{2n\sqrt{n}}
    \]
    By theorem \(3.28\), \(\sum\frac{1}{n^p}\) converges if \(p > 1\) and
    diverges if \(p\leq 1\).
    Therefore,
    \[
    \sum_{n = 1}^{\infty}\frac{1}{2n\sqrt{n}} < \infty
    \]
    since \(p = 3/2\) so
    \[
    \sum_{n = 1}^{\infty}\frac{1}{n\bigl(\sqrt{n + 1} + \sqrt{n}\bigr)} <
    \infty.
    \]
  \item
    \(a_n = \bigl(\sqrt[n]{n} - 1\bigr)^n\)
    \par\smallskip
    By the root test,
    \[
    \lim_{n\to\infty}\sqrt[n]{\lvert\sqrt[n]{n} - 1\rvert^n} =
    \lim_{n\to\infty}\lvert\sqrt[n]{n} - 1\rvert.
    \]
    Let \(x_n = \sqrt[n]{n} - 1\).
    Then
    \[
    n = (x_n - 1)^n = \sum_{k = 0}^n\binom{n}{k}x_n^k = 1 + nx_n +
    \frac{n(n-1)}{2}x_n^2 + \cdots
    \]
    so \(\frac{n(n-1)}{2}x_n^2 < n\Rightarrow x_n^2 < \frac{2}{n - 1}
    \Rightarrow x_n < \sqrt{\frac{2}{n - 1}}\).
    \begin{align*}
      \lim_{n\to\infty}\lvert\sqrt[n]{n} - 1\rvert
      & = \lim_{n\to\infty}\lvert x_n\rvert\\
      & < \lim_{n\to\infty}\sqrt{\frac{2}{n - 1}}\\
      & = 0
    \end{align*}
    Since \(\sqrt{\frac{2}{n - 1}}\) converges and
    \(x_n < \sqrt{\frac{2}{n - 1}}\), \(x_n\) converges.
    By the root and comparison test,
    \[
    \sum_{n = 1}^{\infty}\bigl(\sqrt[n]{n} - 1\bigr)^n
    \]
    converges.
  \item
    \(a_n = \frac{1}{1 + z^n}\) for complex values of \(z\).
    \[
    \sum_{n = 0}^{\infty}\frac{1}{1 + z^n}\leq
    \sum_{n = 0}^{\infty}\frac{1}{z^n}
    \]
    and \(\sum_{n = 0}^{\infty}\frac{1}{z^n}\) converges for
    \(\bigl\lvert\frac{1}{z}\bigr\rvert < 1\).
  \end{enumerate}
\item
  Prove that the convergence of \(\sum a_n\) implies the convergence of
  \[
  \sum\frac{\sqrt{a_n}}{n},
  \]
  if \(a_n\geq 0\).
  \par\smallskip
  For \(a_n\geq 1\), \(\sqrt{a_n}\leq a_n\).
  By the comparison test,
  \[
  \sum\frac{\sqrt{a_n}}{n} < \sum a_n < \infty
  \]
  so for \(a_n\geq 1\), the series converges.
  For \(0\leq a_n < 1\), \(a_n\leq\sqrt{a_n}\).
  We can write all rationals and irrational numbers in \([0,1)\) as \(b/n\)
  for \(b\in\mathbb{R}^{\geq 0}\).
  Then
  \[
  \sum\frac{\sqrt{a_n}}{n} = \frac{\sqrt{b}}{n\sqrt{n}} < \infty
  \]
  since \(p = 3/2 > 1\) so the series converges.
\item
  If \(\sum a_n\) converges, and if \(\{b_n\}\) is monotonic and bounded, prove that \(\sum a_nb_n\) converges.
  \par\smallskip
  By theorem \(3.14\), we know that monotonic bounded sequences converge.
  Let \(\{b_n\}\to M\) for some number \(M < \infty\) or
  \(\lvert b_n\rvert\leq M\).
  Since \(\sum a_n\) converges, for a given \(\epsilon > 0\) and \(k\geq N\),
  \(m\geq k\geq N\) implies that
  \[
  \Bigl\lvert\sum_{n = k}^ma_n\Bigr\rvert\leq\sum_{n = k}^m\lvert a_n\rvert\leq
  \epsilon.
  \]
  Take \(\epsilon = \frac{\epsilon}{M}\).
  Then
  \[
  \Bigl\lvert\sum a_nb_n\Bigr\rvert\leq\sum\lvert a_n\rvert\lvert b_n\rvert\leq
  \sum\lvert a_n\rvert M.
  \]
  Since \(\sum\lvert a_n\rvert\leq\epsilon/M\), the result follows; that is,
  \[
  \sum\lvert a_n\rvert M\leq\epsilon
  \]
  so \(\sum a_nb_n\) converges.
\item
  Find the radius of convergence of each of the following power series:
  \begin{enumerate}[label = (\alph*)]
  \item
    \(\sum n^3z^n\)
    \par\smallskip
    Here will use the ratio test.
    \[
    \limsup_{n\to\infty}\Bigl\lvert\frac{(n + 1)^3z^{n + 1}}{n^3z^n}\Bigr\rvert
    = \lvert z\rvert\limsup_{n\to\infty}\Bigl\lvert\frac{n + 1}{n}\Bigr\rvert^3
    = \lvert z\rvert\limsup_{n\to\infty}\Bigl\lvert\frac{n}{n}\Bigr\rvert^3 =
    \lvert z\rvert
    \]
    Then \(\limsup = \alpha\) and \(R = \frac{1}{\alpha}\) so \(R = 1\) and
    \(\lvert z\rvert < 1\) for convergence.
  \item
    \(\sum\frac{2^n}{n!}z^n\)
    \par\smallskip
    Again, we use the ratio test.
    \[
    \limsup_{n\to\infty}\Bigl\lvert
    \frac{2^{n + 1}z^{n + 1}n!}{2^nz^n(n + 1)!}\Bigr\rvert =
    2\lvert z\rvert\limsup_{n\to\infty}\Bigl\lvert\frac{1}{n + 1}\Bigr\rvert =
    0
    \]
    Thus, \(R = \infty\).
  \item
    \(\sum\frac{2^n}{n^2}z^n\)
    \par\smallskip
    Following the same test, we have
    \[
    \limsup_{n\to\infty}\Bigl\lvert
    \frac{2^{n + 1}z^{n + 1}n^2}{2^nz^n(n + 1)^2}\Bigr\rvert =
    2\lvert z\rvert\limsup_{n\to\infty}\Bigl\lvert\frac{n}{n}\Bigr\rvert^2 =
    2\lvert z\rvert
    \]
    so \(R = 1/2\) and \(\lvert z\rvert < 1/2\) for convergence.
  \item
    \(\sum\frac{n^3}{3^n}z^n\)
    \[
    \limsup_{n\to\infty}\Bigl\lvert
    \frac{(n + 1)^3z^{n + 1}3^n}{3^{n + 1}z^nn^3}\Bigr\rvert =
    \frac{\lvert z\rvert}{3}
    \limsup_{n\to\infty}\Bigl\lvert\frac{n}{n}\Bigr\rvert^3 =
    \frac{\lvert z\rvert}{3}
    \]
    so \(R = 3\) and \(\lvert z\rvert < 3\) for convergence.
  \end{enumerate}
\item
  Suppose the the coefficients of the power series \(\sum a_nz^n\) are
  integers, infinitely many of which are distinct from zero.
  Prove that the radius of convergence is at most \(1\).
  \par\smallskip
\item
  Suppose \(a_n > 0\), \(s_n = a_1 + \cdots + a_n\), and \(\sum a_n\) diverges.
  \begin{enumerate}[label = (\alph*)]
  \item
    Prove that \(\sum\frac{a_n}{1 + a_n}\) diverges.
  \item
    Prove that
    \[
    \frac{a_{N + 1}}{s_{N + 1}} + \cdots + \frac{a_{N + k}}{s_{N + k}}\geq 1
    - \frac{s_N}{s_{N + k}}
    \]
    and deduce that \(\sum\frac{a_n}{s_n}\) diverges.
  \item
    Prove that
    \[
    \frac{a_n}{s_n^2}\leq \frac{1}{s_{n - 1}} - \frac{1}{s_n}
    \]
    and deduce that \(\sum\frac{a_n}{s_n^2}\) converges.
  \item
    What can be said about
    \[
    \sum\frac{a_n}{1 + na_n}\quad\text{and}\quad\sum\frac{a_n}{1 + n^2a_n}
    \mbox{?}
    \]
  \end{enumerate}
\item
  Suppose \(a_n > 0\) and \(\sum a_n\) converges.
  Put
  \[
  r_n = \sum_{m = n}^{\infty}a_m.
  \]
  \begin{enumerate}[label = (\alph*)]
  \item
    Prove that
    \[
    \frac{a_m}{r_m} + \cdots + \frac{a_n}{r_n} > 1 - \frac{r_n}{r_m}
    \]
    if \(m < n\), and deduce that \(\sum\frac{a_n}{r_n}\) diverges.
  \item
    Prove that
    \[
    \frac{a_n}{\sqrt{r_n}} < 2\bigl(\sqrt{r_n} - \sqrt{r_{n + 1}}\bigr)
    \]
    and deduce that \(\sum\frac{a_n}{\sqrt{r_n}}\) converges.
  \end{enumerate}
\end{enumerate}
%%% Local Variables:
%%% mode: latex
%%% TeX-master: t
%%% End:
