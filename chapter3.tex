\chapter{Numerical Sequences and Series}

\begin{enumerate}
\item
  Prove that convergence of \(\{s_n\}\) implies convergence of
  \(\{\lvert s_n\rvert\}\).
  Is the converse true?
  \par\smallskip
  Since \(\{s_n\}\) converges, it is Cauchy.
  Let \(\epsilon > 0\) be given.
  There exist \(n,m > N\) such that \(\lvert s_n - s_m\rvert < \epsilon\) since
  \(\{s_n\}\) is Cauchy.
  \begin{align*}
    \lvert s_n\rvert & = \lvert s_n - s_m + s_m\rvert\\
                     & \leq \lvert s_n - s_m\rvert + \lvert s_m\rvert\\
    \lvert s_n\rvert - \lvert s_m\rvert & \leq \lvert s_n - s_m\rvert\\
    \intertext{Similarly, we can show}
    \lvert s_m\rvert - \lvert s_n\rvert & \leq \lvert s_m - s_n\rvert\\
                     & = \lvert s_n - s_m\rvert
  \end{align*}
  so
  \[
  \bigl\lvert\lvert s_n\rvert - \lvert s_m\rvert\bigr\rvert\leq
  \lvert s_n - s_m\rvert < \epsilon.
  \]
  No.
  Consider the sequence \(\{s_n\} = (-1)^n\).
  Let \(\epsilon = 1\).
  If \(\{s_n\}\) converges, it will converge to \(\pm 1\).
  WLOG assume \(s_n\to 1\).
  Let \(n > N\) such that \(n\) is odd.
  \[
  \lvert (-1)^n - 1\rvert = \lvert -1 - 1\rvert = 2\not < \epsilon
  \]
  Therefore, the sequence \(\{s_n\}\) doesn't converge.
  However, \(\{\lvert s_n\rvert\}\) does converge to \(1\).
  Let \(\epsilon > 0\) given.
  There exist an \(n > N\) such that
  \(\bigl\lvert\lvert (-1)^n\rvert - 1\bigr\rvert < \epsilon\).
  For any \(n > N\), \((-1)^n = \pm 1\) and \(\lvert \pm 1\rvert = 1\).
  \[
  \bigl\lvert\lvert (-1)^n\rvert - 1\bigr\rvert =
  \lvert 1 - 1\rvert = 0 < \epsilon
  \]
\item
  Caclulate \(\lim_{n\to\infty}\sqrt{n^2 + n} - n\).
  \begin{align*}
    \lim_{n\to\infty}\sqrt{n^2 + n} - n
    & = \lim_{n\to\infty}\Bigl(\sqrt{n^2 + n} - n\Bigr)
      \frac{\sqrt{n^2 + n} + n}{\sqrt{n^2 + n} + n}\\
    & = \lim_{n\to\infty}\frac{n}{\sqrt{n^2 + n} + n}\\
    & = \lim_{n\to\infty}\frac{n}{\sqrt{n^2 + n} + n}\frac{1/n}{1/n}\\
    & = \lim_{n\to\infty}\frac{1}{\sqrt{1 + 1/n} + 1}\\
    & = \frac{1}{2}
  \end{align*}
\item
  If \(s_1 = \sqrt{2}\) and
  \[
  s_{n + 1} = \sqrt{2 + \sqrt{s_n}}
  \]
  \(n\in\mathbb{Z}^+\) prove that \(\{s_n\}\) converges, and that \(s_n < 2\)
  for \(n\in\mathbb{Z}^+\).
  \par\smallskip
  Let \(s_{n + 1} = \sqrt{2 + \sqrt{s_n}}\) be written as
  \[
  s = \sqrt{2 + s}\Rightarrow s^2 - s - 2 = 0.
  \]
  Then \(\sqrt{2 + \sqrt{s_n}} < \sqrt{2 + s}\).
  Since we are dealing with real numbers, we are only looking for positive
  \(s\).
  \[
  s^2 - s - 2 = (s - 2)(s + 1) = 0\eqnumtag\label{3.2}
  \]
  so \(s = 2, -1\).
  Thus, \(s_n < 2\) so \(\{s_n\}\) is bounded above by two.
  Additionally, since \(s_1 = \sqrt{2}\), we have that \(\sqrt{2} < s_n < 2\).
  The parabola is concave up and symmetrical about \(s = 1/2\).
  That is, \(s\) monotonically increases from \((1/2, \infty)\) so \(\{s_n\}\)
  monotonoically increases on \(\bigl[\sqrt{2}, 2\bigr)\).
  Therefore, \(\{s_n\}\) converges and it converges to \(2\).
\item
  Find the upper and lower limits of the sequence \(\{s_n\}\) defined by
  \[
  s_1 = 0,\qquad s_{2m} = \frac{s_{2m - 1}}{2},\qquad
  s_{2m + 1} = \frac{1}{2} + s_{2m}.
  \]
  Let's determine a few of the terms.
  Then \(s_{2m} = \{0,1/4,3/8,7/16,\ldots\}\) and
  \(s_{2m + 1} = \{1/2,3/4,7/8,15/16,\ldots\}\) or we can write them as
  \begin{align*}
    s_{2m} & = \frac{1}{2} - \frac{1}{2^m}\\
    s_{2m + 1} & = 1 - \frac{1}{2^m}
  \end{align*}
  The \(\lim_{n\to\infty}\sup s_n = \lim_{n\to\infty} 1 - \frac{1}{2^n} = 1\)
  and
  \(\lim_{n\to\infty}\inf s_n = \lim_{n\to\infty} \frac{1}{2} - \frac{1}{2^n}
  = \frac{1}{2}\).
\item
  For any two real sequences \(\{a_n\}\), \(\{b_n\}\), prove that
  \[
  \lim_{n\to\infty}\sup (a_n + b_n)\leq\lim_{n\to\infty}\sup a_n +
  \lim_{n\to\infty}\sup b_n,
  \]
  provided the sum on the right is not of the form \(\infty - \infty\).
\end{enumerate}
%%% Local Variables:
%%% mode: latex
%%% TeX-master: t
%%% End:
