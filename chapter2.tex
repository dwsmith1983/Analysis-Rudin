\section{Basic Topology}

\begin{enumerate}
\item
  Prove that the empty set is a subset of every set.
  \par\smallskip
  Let \(A\) be set.
  If \(x\not\in A\), then \(x\not\in\varnothing\).
  Since \(\varnothing\) is the empty set, \(x\not\in\varnothing\) is a given.
  By contrapositive, if \(x\in\varnothing\), then \(x\in A\); therefore,
  \(\varnothing\subset A\).
\item
  A complex number \(z\) is said to be \textit{algebraic} if there are integers
  \(a_0,\ldots,a_n\), not all zero, such that
  \[
  a_0z^n + a_1z^{n - 1} + \cdots + a_{n - 1}z + a_n = 0.
  \]
  Prove that the set of all algebraic numbers is countable.
  \textit{Hint: For every positive integer \(N\) there are only finitely many
    equations with}
  \[
  n + \lvert a_0\rvert + \lvert a_1\rvert + \cdots + \lvert a_n\rvert = N.
  \]
  Let \(N\in\mathbb{Z}^+\) and \(A_N\) be the set of algebraic equations for a
  given \(N\).
  Since \(1\leq n\leq N\), each \(A_N\) is finite.
  The set of algebraic numbers is \(\bigcup_{N\in\mathbb{Z}^+}A_n\).
  The union of countable sets is countable so the set of algebraic numbers is
  countable.
\item
  Prove that there exist real numbers which are not algebraic.
  \par\smallskip
  The set of algebraic numbers are countable.
  Therefore, the set of algebraic real numbers would also be countable.
  The real numbers are an uncountable set and the union of uncontable sets are
  not contable. We have reached a contradiction so there are real numbers which
  are not algebraic.
\item
  Is the set of all irrational real numbers countable?
  \par\smallskip
  No.
  Let \(\mathbb{I}\) be the set of irrational numbers and \(\mathbb{Q}\) be the
  set of rational numbers.
  Then \(\mathbb{R} = \mathbb{I}\cup\mathbb{Q}\).
  The set of rational numbers is countable.
  % prove Q is countable
  If \(\mathbb{I}\) were countable, then \(\mathbb{R}\) would be countable as
  well.
\item
  Construct a bounded set of real numbers with exactly three limit points.
  \par\smallskip
  Let \(A_0 = \{1/n\mid n\in\mathbb{Z}^+\}\),
  \(A_1 = \{1 + 1/n\mid n\in\mathbb{Z}^+\}\), and
  \(A_2 = \{2 + 1/n\mid n\in\mathbb{Z}^+\}\).
  Then the limit point of \(A_1\) is \(0\), the limit point of \(A_2\) is
  \(1\), and the limit point of \(A_2\) is \(2\).
  Let \(S = A_1\cup A_2\cup A_3\).
  Now \(S\) is bounded below by zero and above by three with limit points
  \(0,1,2\).
\item
  Let \(E'\) be the set of all limit points of a set \(E\).
  Prove that \(E'\) is closed.
  Prove that \(E\) and \(\bar{E}\) have the same limit points.
  (Recall that \(\bar{E} = E\cup E'\).)
  Do \(E\) and \(E'\) always have the same limit points?
  \par\smallskip
  Let \(x\in X\setminus E'\) and \(x'\in E'\).
  Then \(x\) is not a limit point of \(E\).
  Let \(\epsilon > 0\) be given and \(y\in E\).
  Set \(\epsilon = \min\bigl\{\lvert d(x,x') - d(x,y)\rvert,
  \lvert d(y,x') - d(x,y)\rvert\bigr\}\).
  Then \(N_{\epsilon}(x) = d(x,y) < \epsilon\) is a neighborhood of \(x\) that
  contains \(y\neq x\) such that \(y\not\in E'\).
  Therefore, \(x\) is an interior point of \(X\setminus E'\) so
  \(X\setminus E'\) is open and its complement is closed, namely \(E'\).
  Let \(x'\) be a limit point of \(E\).
  Then \(x'\) is a limit point of \(\bar{E}\) since \(\bar{E} = E\cup E'\).
  Now, let \(x'\) be a limit point of \(\bar{E}\).
  Since \(x'\in\bar{E}\), we can construct a neighborhood of \(x'\) such that
  for \(y\neq x'\) for \(y\in\bar{E}\).
  If \(y\in E\), \(x'\) is a limit point of \(E\) and \(E'\).
  Suppose \(y\in\bar{E}\setminus E\).
\item
  Let \(A_1, A_2, \ldots\) be subsets of a metric space.
  \begin{enumerate}[label = (\alph*)]
  \item
    If \(B_n = \bigcup_{i = 1}^nA_i\), prove that
    \(\bar{B}_n = \bigcup_{i = 1}^n\bar{A}_i\) for \(n = 1, 2, \ldots\)
  \item
    If \(B_n = \bigcup_{i = 1}^nA_i\), prove that
    \(\bar{B}\supset\bigcup_{i = 1}^n\bar{A}_i\).
  \end{enumerate}
\item
  Is every point of every open set \(E\subset\mathbb{R}^2\) a limit point of
  \(E\)?
  Answer the same question for closed sets of \(\mathbb{R}^2\).
\item
  Let \(E^{\circ}\) denote the set of all interior points of a set \(E\).
  \begin{enumerate}[label = (\alph*)]
  \item
    Prove that \(E^{\circ}\) is always open.
  \item
    Prove that \(E\) is open if and only if \(E^{\circ} = E\).
  \item
    If \(G\subset E\) and \(G\) is open, prove that \(G\subset E^{\circ}\).
  \item
    Prove that the complement of \(E^{\circ}\) is the closure of the
    complement of \(E\).
  \item
    Do \(E\) and \(\bar{E}\) always have the same interiors?
  \item
    Do \(E\) and \(E^{\circ}\) always have the same closures?
  \end{enumerate}
\item
  Let \(X\) be an infinite set.
  For \(p\in X\) and \(q\in X\), define
  \[
  d(p, q) =
  \begin{cases}
    1, & \text{if } p\neq q\\
    0, & \text{if } p = q
  \end{cases}
  \]
  Prove that this a metric space.
  Which subsets of the resulting metric space are open?
  Which are closed? Which are compact?
\item
  For \(x\in\mathbb{R}\) and \(y\in\mathbb{R}\), define
  \begin{align*}
    d_1(x, y) & = (x - y)^2\\
    d_2(x, y) & = \sqrt{\lvert x - y\rvert}\\
    d_3(x, y) & = \lvert x^2 - y^2\rvert\\
    d_4(x, y) & = \lvert x - 2y\rvert\\
    d_5(x, y) & = \frac{\lvert x - y\rvert}{1 + \lvert x - y\rvert}
  \end{align*}
  Determine for each of these, whether it is a metric or not.
\item
  Let \(K\subset\mathbb{R}\) consist of \(0\) and the numbers \(1/n\) for
  \(n = 1, 2, \ldots\)
  Prove that \(K\) is compact directly from the definition (without using the
  Heine-Borel theorem).
\item
  Construct a compact set of real numbers whose limit points form a countable
  set.
\item
  Give an example of an open cover of the segment \((0, 1)\) which has no
  finite subcover.
\item
  Show that Theorem \(2.36\) and its Corollary become false (in \(\mathbb{R}\),
  for example) if the word "compact" is replaced by "closed" or by "bounded".
\item
  Regard \(\mathbb{Q}\), the set of all rational numbers, as a metric space,
  with \(d(p, q) = \lvert p - q\rvert\).
  Let \(E\) be the set of all \(p\in\mathbb{Q}\) such that \(2 < p^2 < 3\).
  Show that \(E\) is closed and bounded in \(\mathbb{Q}\), but that \(E\) is
  not compact.
  Is \(E\) open in \(\mathbb{Q}\)?
\item
  Let \(E\) be the set of all \(x\in[0, 1]\) whose decimal expansion contains
  only the digits \(4\) and \(7\).
  Is \(E\) countable?
  Is \(E\) dense in \([0, 1]\)?
  Is \(E\) compact?
  Is \(E\) perfect?
\item
  Is there a nonempty perfect set in \(\mathbb{R}\) which contains no rational
  number?
\item
  \begin{enumerate}[label = (\alph*), ref = \theenumi{} (\alph*)]
  \item
    If \(A\) and \(B\) are disjoint closed sets in some metric space \(X\),
    prove that they are separated.
  \item
    Prove the same for disjoint open sets.
  \item
    \label{19c}
    Fix \(p\in X\), \(\delta > 0\), define \(A\) to be the set of all
    \(q\in X\) for which \(d(p, q) < \delta\), define \(B\) similarly, with
    \(>\) in place of \(<\).
    Prove that \(A\) and \(B\) are separated.
  \item
    Prove that every connected metric space with at least two points is
    uncountable.
    \textit{Hint: Use \cref{19c}.}
  \end{enumerate}
\item
  Are closures and interiors of connected sets always connected?
  (Look at subsets of \(\mathbb{R}^2\).)
\item
  Let \(A\) and \(B\) be separated subsets of some \(\mathbb{R}^k\), suppose
  \(\mathbf{a}\in A\), \(\mathbf{b}\in B\), and define
  \[
  \mathbf{p}(t) = (1 - t)\mathbf{a} + t\mathbf{b}
  \]
  for \(t\in\mathbb{R}\).
  Put \(A_0 = \mathbf{p}^{-1}(A)\), \(B_0 = \mathbf{p}^{-1}(B)\).
  (Thus \(t\in A_0\) if and only if \(\mathbf{p}(t)\in A\).)
  \begin{enumerate}[label = (\alph*)]
  \item
    Prove that \(A_0\) and \(B_0\) are separated subsets of \(\mathbb{R}\).
  \item
    Prove that there exists \(t_0\in (0, 1)\) such that
    \(\mathbf{p}(t_0)\not\in A\cup B\).
  \item
    Prove that every convex subset of \(\mathbf{R}^k\) is connected.
  \end{enumerate}
\item
  A metric space is \textit{separable} if it contains a countable dense subset.
  Show that \(\mathbb{R}^k\) is separable.
  \textit{Hint: Consider the set of points which have only rational
    coordinates}
\item
  A collection \(\{V_{\alpha}\}\) of open subsets of \(X\) is said to be a
  \textit{base} for \(X\) if the following is true: For every \(x\in X\) and
  every open set \(G\subset X\) such that \(x\in G\), we have
  \(x\in V_{\alpha}\subset G\) for some \(\alpha\).
  In other words, every open set in \(X\) is the union of a subcollection of
  \(\{V_{\alpha}\}\).
  Prove that every separable metric space has a \textit{countable} base.
  \textit{Hint: Take all neighborhoods with rational radius and center in some
    countable dense subset of \(X\).}
\item
  Let \(X\) be a metric space in which every infinite subset has a limit point.
  Prove that \(X\) is separable.
  \textit{Hint: Fix \(\delta > 0\), and pick \(x_1\in X\).}
  Having chosen \(x_1,\ldots,x_j\in X\), choose \(x_{j + 1}\in X\), if
  possible, so that \(d(x_i, x_{j + 1})\geq\delta\) for \(i = 1,\ldots,j\).
  Show that this process must stop after a finite number of steps, and that
  \(X\) can therefore be covered by finitely many neighborhoods of radius
  \(\delta\).
  Take \(\delta = 1/n\) \((n = 1,2,\ldots)\), and consider the centers of the
  corresponding neighborhoods.
\item
  Prove that every compact metric space \(K\) has a countable base, and that
  \(K\) is therefore separable.
  \textit{Hint: For every positive integer \(n\), there are finitely many
    neighborhoods of radius \(1/n\) whose union covers \(K\).}
\item
  Let \(X\) be a metric space in which every infinite subset has a limit point.
  Prove that \(X\) is compact.
  \textit{Hint: By exercise \(23\) and \(24\), \(X\) has a countable base.}
  It follows that every open cover of \(X\) has a countable subcover
  \(\{G_n\}\), \(n = 1,2,\ldots\)
  If no finite subcollection of \(\{G_n\}\) covers \(X\), then the complement
  \(F_n\) of \(G_1\cup\cdots\cup G_n\) is nonempty for each \(n\), but
  \(\bigcap F_n\) is empty.
  If \(E\) is a set which contains a point from each \(F_n\), consider a limit
  point of \(E\), and obtain a contradiction.
\item
  Define a point \(p\) in a metric space \(X\) to be
  \textit{condensation point} of a set \(E\subset X\) if every neighborhood of
  \(p\) contains uncountably many points of \(E\).
  Suppose \(E\subset\mathbb{R}^k\), \(E\) is uncountable, and let \(P\) be the
  set of all condensation points of \(E\).
  Prove that \(P\) is perfect and that at most countably many points of \(E\)
  are not in \(P\).
  In other words, show that \(P^c\cap E\) is at most countable.
  \textit{Hint: Let \(\{V_n\}\) be a countable base of \(\mathbb{R}^k\), let
    \(W\) be the union of those \(V_n\) for which \(E\cap V_n\) is at most
    countable, and show that \(P = W^c\).}
\item
  Prove that every closed set in a separable metric space is the union of a
  (possibly empty) perfect set and a set which is at most countable.
  \textit{(Corollary: Every countable close set in \(\mathbb{R}^k\) has
    isolated points.)
    Hint: Use exercise \(27\).}
\item
  Prove that every open set in \(\mathbb{R}\) is the union of an at most
  countable collection of disjoints segments.
  \textit{Hint: Use exercise \(22\).}
\item
  Imitate the proof of Theorem \(2.43\) to obtain the following results:
  \par\smallskip
  If \(\mathbb{R}^k = \bigcup_{n = 1}^{\infty}F_n\), where each \(F_n\) is a
  closed subset of \(\mathbb{R}^k\), then at least one \(F_n\) has a nonempty
  interior.
  \par\smallskip
  \textit{Equivalent statement:} If \(G_n\) is a dense open subset
  \(\mathbb{R}^k\), for \(n = 1,2,\ldots\) then
  \(\bigcap_{n = 1}^{\infty}G_n\) is not empty (in fact, it is dense in
  \(\mathbb{R}^k\)).
  \par\smallskip
  (This is a special case of Baire's theorem; see exercise \(22\), chapter
  \(3\), for the general case.)
\end{enumerate}

%%% Local Variables:
%%% mode: latex
%%% TeX-master: t
%%% End:
